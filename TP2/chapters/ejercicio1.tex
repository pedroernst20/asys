\chapter{Ejercicio 1}

\section{Actividad 1}

\textbf{Actividad 1}. Dada la secuencia de senales {$\varphi_{n} (t)=e^{j(n\omega)t}: n \in \mathbb{Z}$}, con $T = \dfrac{2\pi}{\omega}$. Demostrar:

\subsection{A}

\textbf{A)} El periodo fundamental de la senal $\varphi_{n}(t)$ es $T_n=\dfrac{2\pi}{n\omega}$. ¿Por qué se puede afirmar que la suma entre estas señales está bien definida $\varphi_{n}(t)$?

\textbf{i.} Lo primero que realizaremos seria corroborar que $T_n$ es un periodo.

$$\varphi_{n}(t+T_n)=\varphi_{n}(t)$$

$$e^{jn\omega(t+T_n)}=e^{jn\omega(t)}$$

$$e^{jn\omega(t+\dfrac{2\pi}{n\omega})}=e^{jn\omega(t)}$$

$$e^{jn\omega(t)+jn\omega\dfrac{2\pi}{n\omega}}=e^{jn\omega(t)}$$

$$e^{jn\omega(t)+j2\pi}=e^{jn\omega(t)}$$

$$e^{jn\omega(t)} \cdot e^{j2\pi} = e^{jn\omega(t)}$$

$$e^{jn\omega(t)} \cdot 1 = e^{jn\omega(t)}$$

$$e^{jn\omega(t)} = e^{jn\omega(t)}$$

\textbf{ii.} Para ver que $T_n$ es el periodo fundamental, suponemos que exisite un $T'>0$ con $T'<T_n$, tal que.

$$\varphi_n(t+T')=\varphi_n(t)$$

Entonces $e^{jn\omega T'} = 1$, por lo que existe $k \in \mathbb{Z}$ tal que

$$n\omega T'= 2\pi k$$

Despejando $n\omega$

$$T'=\dfrac{2\pi k}{n\omega}$$

Donde si $k=0$ entonces $T'=0$, lo cual contradic lo que postulamos al principio que $T'>0$

Si $|k| \ge 1$, entonces

$$T' \ge \dfrac{2\pi}{|n|\omega} = T_{|n|}$$

Lo cual contradice que $T'<T_n$. Por lo tanto no existe $T'\in(0,T_n)$ que sea periodo.

\textbf{iii.}

Ahora bien podemos concluir que la suma $\sum_{n \in \mathbb{Z}} \phi_n(t)$ está bien definida porque todas las señales $\phi_n(t)$ son periódicas con un periodo común $T = \frac{2\pi}{\omega}$ (múltiplo entero de $T_n$). Esto garantiza que la superposición de señales mantenga la periodicidad global.

\vspace{0.5cm}

\subsection{B}

\textbf{B)} Calcular $E_n = (\phi_n(t), \phi_n(t))_T \quad \text{y} \quad (\phi_n(t), \phi_m(t))_T, \quad \forall n,m \in \mathbb{Z}$

Como primer paso definiremos
\[
\langle \varphi_n, \varphi_m \rangle_T = \frac{1}{T}\int_0^T \varphi_n(t)\overline{\varphi_m(t)}dt
\]

\textbf{Caso 1: $n = m$}
\begin{align*}
\langle \varphi_n, \varphi_n \rangle_T &= \frac{1}{T}\int_0^T e^{jn\omega t}e^{-jn\omega t}dt \\
&= \frac{1}{T}\int_0^T 1\,dt = 1 \quad \forall n
\end{align*}

\textbf{Caso 2: $n \neq m$}
\begin{align*}
\langle \varphi_n, \varphi_m \rangle_T &= \frac{1}{T}\int_0^T e^{j(n-m)\omega t}dt \\
&= \left.\frac{e^{j(n-m)\omega t}}{j(n-m)\omega T}\right|_0^T = 0 \quad \text{(por periodicidad)}
\end{align*}

\vspace{0.5cm}

\subsection{C}

\textbf{C)} La secuencia de señales $\{ \phi_n(t) = e^{j(n\omega)t} : n \in \mathbb{Z} \}$ es base de Fourier.

\textbf{Propiedades requeridas:}
\begin{enumerate}[label=(\roman*)]
\item \textbf{Ortogonalidad}: $\langle \varphi_n, \varphi_m \rangle_T = \delta_{nm}$ (demostrado en B)
\item \textbf{Completitud}: Cualquier señal $x(t)$ periódica puede expresarse como:
\[
x(t) = \sum_{n=-\infty}^{\infty} C_n \varphi_n(t)
\]
\end{enumerate}

\vspace{0.5cm}

\subsection{D}

\textbf{D)} La secuencia de señales $\{ \rho_n^0(t) = \cos(n\omega t) = \tfrac{1}{2}\phi_n(t) + \tfrac{1}{2}\overline{\phi_n(t)} : n \geq 0 \}$ es base de Fourier.

\textbf{Relación con $\varphi_n(t)$:}
\[
\cos(n\omega t) = \frac{1}{2}\varphi_n(t) + \frac{1}{2}\varphi_{-n}(t)
\]

\textbf{Ortogonalidad:}
\[
\langle \rho_n^0, \rho_m^0 \rangle_T = 
\begin{cases}
1 & n = m = 0 \\
\frac{1}{2} & n = m \neq 0 \\
0 & n \neq m
\end{cases}
\]

\vspace{0.5cm}

\subsection{E}

\textbf{E)} La secuencia de señales \[\{ \rho_n^1(t) = \sin(n\omega t) = \tfrac{1}{2j}\phi_n(t) - \tfrac{1}{2j}\overline{\phi_n(t)} : n > 0 \}\] es base de Fourier.

\textbf{Relación con $\varphi_n(t)$:}
\[
\sin(n\omega t) = \frac{1}{2j}\varphi_n(t) - \frac{1}{2j}\varphi_{-n}(t)
\]

\textbf{Ortogonalidad:}
\[
\langle \rho_n^1, \rho_m^1 \rangle_T = 
\begin{cases}
\frac{1}{2} & n = m \\
0 & n \neq m
\end{cases}
\]
